\documentclass[12pt]{article}
\usepackage[T2A]{fontenc}
\usepackage[utf8]{inputenc}
\usepackage[ukrainian]{babel}
\usepackage{lipsum} % Для генерації випадкового тексту
\usepackage[margin=2cm]{geometry}


\begin{document}

% \noindent запобігає відступу першого рядка, щоб колонки були вирівняні
\noindent
\begin{minipage}[t]{0.48\textwidth}
    \textbf{Варіант 1}

    Розв'яжіть рівняння:

    1) $ 2x + 4 = -6 $

    2) $ 7a - 5 = -12 $

    3) $ 3F + 6 = -9 $

    4) $ \frac{4x}{6} + 3 = 5 $

    5) $ \frac{7a}{3} - 4 = 10 $

    6) $ 1 - \frac{5F}{4} = -14 $

    7) $ 6(5x - 1) - 4 = 20 $

    8) $ 5(2a - 3) + 6 = 1 $
        
    \vspace{1cm} % Вертикальний відступ між варіантами
    
    \textbf{Варіант 3}

    Розв'яжіть рівняння:

    1) $ 5x + 10 = -20 $

    2) $ 4a - 6 = -18 $

    3) $ 7F + 2 = -5 $

    4) $ \frac{6x}{5} + 1 = 13 $

    5) $ \frac{9a}{2} - 5 = 4 $

    6) $ 2 - \frac{3F}{8} = -1 $

    7) $ 3(4x - 2) + 7 = -11 $

    8) $ 6(3a + 5) - 4 = 44 $
    
\end{minipage}
\hfill % Ця команда створює пружний пробіл, що розсовує колонки
\begin{minipage}[t]{0.48\textwidth}
    \textbf{Варіант 2}

    Розв'яжіть рівняння:

    1) $ 4x + 8 = -12 $

    2) $ 6a - 2 = -14 $

    3) $ 2F + 7 = -5 $

    4) $ \frac{5x}{7} + 2 = 12 $

    5) $ \frac{4a}{9} - 7 = 1 $

    6) $ 5 - \frac{2F}{6} = 4 $

    7) $ 2(8x - 3) - 8 = 2 $

    8) $ 4(2a + 5) + 6 = 18 $
        
    \vspace{1cm} % Вертикальний відступ між варіантами

    \textbf{Варіант 4}

    Розв'яжіть рівняння:

    1) $ 3x + 6 = -9 $

    2) $ 5a - 7 = -12 $

    3) $ 8F + 4 = -4 $

    4) $ \frac{7x}{5} + 3 = 17 $

    5) $ \frac{6a}{8} - 2 = 1 $

    6) $ 1 - \frac{4F}{7} = -3 $

    7) $ 5(2x - 1) + 3 = 8 $

    8) $ 3(5a - 2) - 6 = -27 $
    
\end{minipage}

\vspace{1cm}

\noindent
\begin{minipage}[t]{0.48\textwidth}
    \textbf{Варіант 1}

    Розв'яжіть рівняння:

    1) $ 2x + 4 = -6 $

    2) $ 7a - 5 = -12 $

    3) $ 3F + 6 = -9 $

    4) $ \frac{4x}{6} + 3 = 5 $

    5) $ \frac{7a}{3} - 4 = 10 $

    6) $ 1 - \frac{5F}{4} = -14 $

    7) $ 6(5x - 1) - 4 = 20 $

    8) $ 5(2a - 3) + 6 = 1 $
        
    \vspace{1cm} % Вертикальний відступ між варіантами
    
    \textbf{Варіант 3}

    Розв'яжіть рівняння:

    1) $ 5x + 10 = -20 $

    2) $ 4a - 6 = -18 $

    3) $ 7F + 2 = -5 $

    4) $ \frac{6x}{5} + 1 = 13 $

    5) $ \frac{9a}{2} - 5 = 4 $

    6) $ 2 - \frac{3F}{8} = -1 $

    7) $ 3(4x - 2) + 7 = -11 $

    8) $ 6(3a + 5) - 4 = 44 $
    
\end{minipage}
\hfill % Ця команда створює пружний пробіл, що розсовує колонки
\begin{minipage}[t]{0.48\textwidth}
    \textbf{Варіант 2}

    Розв'яжіть рівняння:

    1) $ 4x + 8 = -12 $

    2) $ 6a - 2 = -14 $

    3) $ 2F + 7 = -5 $

    4) $ \frac{5x}{7} + 2 = 12 $

    5) $ \frac{4a}{9} - 7 = 1 $

    6) $ 5 - \frac{2F}{6} = 4 $

    7) $ 2(8x - 3) - 8 = 2 $

    8) $ 4(2a + 5) + 6 = 18 $
        
    \vspace{1cm} % Вертикальний відступ між варіантами

    \textbf{Варіант 4}

    Розв'яжіть рівняння:

    1) $ 3x + 6 = -9 $

    2) $ 5a - 7 = -12 $

    3) $ 8F + 4 = -4 $

    4) $ \frac{7x}{5} + 3 = 17 $

    5) $ \frac{6a}{8} - 2 = 1 $

    6) $ 1 - \frac{4F}{7} = -3 $

    7) $ 5(2x - 1) + 3 = 8 $

    8) $ 3(5a - 2) - 6 = -27 $
    
\end{minipage}

\end{document}